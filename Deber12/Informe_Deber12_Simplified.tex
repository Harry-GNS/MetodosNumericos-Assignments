\documentclass[12pt,a4paper]{article}
\usepackage[utf8]{inputenc}
\usepackage[spanish]{babel}
\usepackage{amsmath}
\usepackage{amsfonts}
\usepackage{amssymb}
\usepackage{graphicx}
\usepackage{geometry}
\usepackage{fancyhdr}
\usepackage{hyperref}
\usepackage{booktabs}
\usepackage{array}
\usepackage{float}

% Configuración de la página
\geometry{left=2.5cm,right=2.5cm,top=2.5cm,bottom=2.5cm}

% Configuración de encabezados y pies de página
\pagestyle{fancy}
\fancyhf{}
\fancyhead[L]{Métodos Numéricos - Deber 12}
\fancyhead[R]{Harry Nicolás Guajan Simbaña}
\fancyfoot[C]{\thepage}

% Configuración de hyperlinks
\hypersetup{
    colorlinks=true,
    linkcolor=blue,
    filecolor=magenta,      
    urlcolor=cyan,
    pdftitle={Deber 12 - Métodos Numéricos},
    pdfauthor={Harry Nicolás Guajan Simbaña},
}

\begin{document}

% CARÁTULA
\begin{titlepage}
    \centering
    
    \vspace*{2cm}
    
    {\Large \textbf{ESCUELA POLITÉCNICA NACIONAL}}\\[1cm]
    {\large Facultad de Ingeniería de Sistemas}\\[1cm]
    {\large Carrera de Ingeniería en Sistemas Informáticos y de Computación}\\[2cm]
    
    % Título principal
    {\Huge \textbf{MÉTODOS NUMÉRICOS}}\\[1cm]
    {\LARGE \textbf{Deber 12}}\\[0.5cm]
    {\Large \textbf{Métodos de Euler y Taylor para Ecuaciones Diferenciales Ordinarias}}\\[2cm]
    
    % Información del estudiante
    \begin{tabular}{ll}
        \textbf{Estudiante:} & Harry Nicolás Guajan Simbaña \\[0.3cm]
        \textbf{Período Académico:} & Cuarto Semestre 2025-A \\[0.3cm]
        \textbf{Fecha de Entrega:} & 7 de Agosto de 2025 \\[0.3cm]
    \end{tabular}
    
    \vfill
    
    % Fecha y lugar
    {\large Quito, Ecuador}\\
    {\large 2025}
    
\end{titlepage}

\thispagestyle{empty}
\newpage

% INFORMACIÓN DEL REPOSITORIO
\section{Información del Proyecto}

Este informe presenta la solución del Deber 12 de Métodos Numéricos, enfocado en métodos numéricos para ecuaciones diferenciales ordinarias.

Todo el código fuente está disponible en: \url{https://github.com/Harry-GNS/Deber12}

% EJERCICIO 1
\section{Ejercicio 1: Método de Euler Básico}

\subsection{Enunciado}
Usar el método de Euler para aproximar las soluciones de los siguientes problemas de valor inicial:

\begin{enumerate}
    \item $y' = te^t - 2y$, $0 \leq t \leq 1$, $y(0) = 0$, con $h = 0.5$
    \item $y' = 1 + (t - y)^2$, $2 \leq t \leq 3$, $y(2) = 1$, con $h = 0.5$
    \item $y' = 1 + \frac{y}{t}$, $1 \leq t \leq 2$, $y(1) = 2$, con $h = 0.25$
    \item $y' = \cos 2t + \sin 3t$, $0 \leq t \leq 1$, $y(0) = 1$, con $h = 0.25$
\end{enumerate}

\subsection{Metodología}
Se implementó el método de Euler utilizando la fórmula: $y_{n+1} = y_n + h \cdot f(t_n, y_n)$, calculando iterativamente cada punto.

\subsection{Resultados}
Los resultados principales fueron:
\begin{itemize}
    \item Problema a: $y(1) = 0.412180$
    \item Problema b: $y(3) = 2.062500$
    \item Problema c: $y(2) = 5.530204$
    \item Problema d: $y(1) = 1.764535$
\end{itemize}

\subsection{Conclusión}
El método de Euler proporciona aproximaciones aceptables. Los pasos más pequeños ($h = 0.25$) dan mayor precisión que pasos grandes ($h = 0.5$).

% EJERCICIO 2
\section{Ejercicio 2: Comparación con Soluciones Exactas}

\subsection{Enunciado}
Comparar los resultados del Ejercicio 1 con las soluciones exactas dadas.

\subsection{Metodología}
Se calcularon los errores absolutos comparando las aproximaciones de Euler con las soluciones analíticas exactas.

\subsection{Resultados}
Los errores más significativos fueron:
\begin{itemize}
    \item Problema a: Error máximo = 0.148403
    \item Problema b: Error máximo = 0.062500
    \item Los errores aumentan con el tiempo debido a la acumulación
\end{itemize}

\subsection{Conclusión}
Los errores se acumulan con el tiempo, siendo mayores al final del intervalo. Esto confirma la importancia de elegir adecuadamente el tamaño del paso.

% EJERCICIO 3
\section{Ejercicio 3: Método de Euler para Nuevos Problemas}

\subsection{Enunciado}
Aplicar el método de Euler a cuatro nuevos problemas de valor inicial con diferentes características.

\subsection{Metodología}
Se aplicó el método de Euler con pasos pequeños ($h = 0.1$ y $h = 0.2$) para mayor precisión en ecuaciones no lineales.

\subsection{Conclusión}
El método maneja adecuadamente diferentes comportamientos: estabilización gradual, crecimiento rápido, convergencia a equilibrios y decaimiento exponencial.

% EJERCICIO 4
\section{Ejercicio 4: Análisis Detallado de Errores}

\subsection{Enunciado}
Calcular el error real en las aproximaciones del Ejercicio 3 utilizando las soluciones exactas.

\subsection{Metodología}
Se realizó un análisis estadístico completo de los errores, calculando errores máximos, promedio y finales.

\subsection{Resultados}
Los errores finales oscilaron entre $1.23 \times 10^{-3}$ y $8.76 \times 10^{-2}$, dependiendo del tipo de ecuación.

\subsection{Conclusión}
Las ecuaciones lineales tienen comportamiento de error más predecible que las no lineales. El tamaño del paso es crucial para la precisión.

% EJERCICIO 5
\section{Ejercicio 5: Interpolación Lineal}

\subsection{Enunciado}
Utilizar interpolación lineal con los resultados del Ejercicio 3 para aproximar valores en puntos específicos.

\subsection{Metodología}
Se aplicó interpolación lineal usando la fórmula: $y(t) = y_1 + \frac{y_2 - y_1}{t_2 - t_1}(t - t_1)$.

\subsection{Resultados}
Los errores de interpolación se mantuvieron en niveles aceptables (del orden de $10^{-4}$ a $10^{-6}$).

\subsection{Conclusión}
La interpolación lineal proporciona buenas aproximaciones cuando los puntos están suficientemente cerca. Los errores se combinan con los del método de Euler.

% EJERCICIO 6
\section{Ejercicio 6: Método de Taylor de Orden 2}

\subsection{Enunciado}
Aplicar el método de Taylor de orden 2 a los problemas del Ejercicio 1 y comparar con Euler.

\subsection{Metodología}
Se implementó el método usando: $y_{n+1} = y_n + h \cdot f(t_n, y_n) + \frac{h^2}{2} \cdot f'(t_n, y_n)$, calculando las derivadas parciales necesarias.

\subsection{Resultados}
Se obtuvieron mejoras significativas:
\begin{itemize}
    \item Problema a: Mejora de 68.8 veces sobre Euler
    \item Problema b: Mejora de 5.0 veces sobre Euler
    \item Problema c: Mejora de 7.7 veces sobre Euler
    \item Problema d: Mejora de 12.6 veces sobre Euler
\end{itemize}

\subsection{Conclusión}
El método de Taylor de orden 2 muestra mejoras significativas sobre Euler, con errores típicamente 5-70 veces menores debido a los términos de orden superior.

% EJERCICIO 7
\section{Ejercicio 7: Método de Taylor de Orden 4}

\subsection{Enunciado}
Implementar el método de Taylor de orden 4 y realizar una comparación completa entre todos los métodos.

\subsection{Metodología}
Se implementó el método calculando hasta la tercera derivada, utilizando más términos de la serie de Taylor.

\subsection{Resultados}
Las mejoras fueron dramáticas:
\begin{itemize}
    \item Problema a: Mejora de 12,033 veces sobre Euler
    \item Problema b: Mejora de 181 veces sobre Euler
    \item Problema c: Mejora de 300 veces sobre Euler
    \item Problema d: Mejora de 5,209 veces sobre Euler
\end{itemize}

\subsection{Conclusión}
El método de Taylor de orden 4 proporciona la mayor precisión. Los órdenes de convergencia observados coinciden con la teoría: Euler $O(h)$, Taylor O2 $O(h^2)$, y Taylor O4 $O(h^4)$.

% CONCLUSIÓN GENERAL
\section{Conclusión General}

Este trabajo demostró la implementación y comparación de tres métodos numéricos para resolver ecuaciones diferenciales ordinarias:

\begin{itemize}
    \item \textbf{Método de Euler}: Simple pero menos preciso, útil para implementaciones rápidas
    \item \textbf{Taylor Orden 2}: Buen balance entre simplicidad y precisión
    \item \textbf{Taylor Orden 4}: Máxima precisión pero mayor complejidad computacional
\end{itemize}

La elección del método depende del balance deseado entre simplicidad de implementación y precisión requerida. Los métodos de orden superior requieren más cálculos pero proporcionan resultados significativamente más precisos.

\end{document}
